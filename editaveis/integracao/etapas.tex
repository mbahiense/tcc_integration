\chapter[Middleware: GSAN e Asterisk]{\textbf{\uppercase{Middleware: Gsan e Asterisk}}}
%\addcontentsline{toc}{chapter}{Processo de Integração}

\textit{Neste capítulo são apresentados os procedimentos para a integração dos sistemas envolvidos, expondo o detalhes de implementação e configuração da solução proposta.}


\section{\uppercase{Arquitetura da Solução}}

Após o estudo sobre o sistema GSAN e software Asterisk foram identificadas diversas formas de realizar a integração, entre os sistemas pelo fato da existência de vários protocolos possíveis de comunicação, no entanto a solução adotada será visando a reusabilidade, baixo custo de manutenção e o uso de tecnologias que já tenham uma maturidade no mercado. Foram escolhidos os protocolos SOAP\footnote{SOAP - \textit{Simple Object Access Protocol}} e AGI\footnote{AGI - \textit{Asterisk Gateway Interface}} para serem implementados por um \textit{Middleware}, este é responsável em assumir o papel de intermediário entre os sistemas GSAN e Asterisk. O Middleware faz uso do protocolo SOAP para se comunicar com o sistema GSAN, já para a comunicação com o sistema \textit{Asterisk} utiliza a interface de comunicação AGI, dessa forma o \textit{Middleware} abstrai as disparidades entre as plataformas dos sistemas. No sistema Asterisk foi implementado o recurso chamado URA, Unidade de Resposta Audível tornando possível programar um fluxo pré-determinado de atendimento das solicitações recepcionadas por chamadas telefônicas.

Com o intuito de facilitar o entendimento da comunicação entre os sistemas, a seguir será exposto o diagrama de implantação da solução descrita acima, explanado os principais detalhes adotados nesta integração:


\begin{figure}[H]
	\centering
	\caption{\textbf{Diagrama de implantação da solução}}
	\label{figura:diagramaImplantacao}
	\begin{subfigure}[H]{\textwidth}
		\centering
		\includegraphics{figuras/diagrama_implantacao.png}
		\legend {\fontsize{10}{12}\selectfont {Fonte: Autoria Própria}.}	
	\end{subfigure}
\end{figure}


Conforme ilustrado acima na figura \ref{figura:diagramaImplantacao}, o sistema GSAN prover uma interface de serviços na forma de \textit{Webservices} utilizando o protocolo de comunicação SOAP\footnote{SOAP - \textit{Simple Object Access Protocol}}, tais serviços são consumidos através do \textit{Middleware} intermediário que também prover uma interface AGI de serviços, para então serem consumidos pelo Asterisk e posteriormente responder as solicitações da Unidade de Resposta Audível.

O processo de integração é composto por tecnologias com paradigmas diferenciados, no entanto para melhorar o entendimento dos detalhes de compatibilidades adotados segue abaixo a tabela \ref{tabela:tecnologiasUtilizadas} das tecnologias utilizadas e versões correspondentes.


\begin{table}[H]
	\footnotesize
	\caption{\textbf{Tecnologias utilizadas}}
	\label{tabela:tecnologiasUtilizadas}
	\begin{tabular}{|p{4cm}|p{7cm}|p{2cm}|}
		\hline
		\textbf{Software} & \textbf{Finalidade} & \textbf{Versão} \\
		\hline
		Java Plataform, \textit{Enterprise Edition} 5 (JEE) & Conjunto de Tecnologias e Serviços para implementar soluções da Plataforma Java com estabilidade, segurança e escalabilidade. & 5 \\
		\hline
		JBoss 				& Servidor de aplicação que implementa especificações JEE. 								& 4.0.1 sp1 \\
		\hline
		Hibernate 			& Framework utilizado para fazer o mapeamento objeto-relacional. É responsável pela camada de persistência. & 3.1 \\
		\hline
		PostgreSQL 			& Banco de dados relacional. 															& 9.3.4 \\
		\hline
		Apache Ant 			& Geração de \textit{Enterprise Application Resources} (EAR) deploy’s. 					& 1.6.2 \\
		\hline
		JasperReports 		& Tecnologia utilizada para criação de relatórios em PDF, HTML, XLS, CSV e XM.L 		& 1.2.2 \\
		\hline
		Struts 				& Framework para controle de navegação e validação Web. 								& 1.1	 \\
		\hline
		Disc-OS 			& Distribuição CentOS 2.6 que implementa interface web para a ferramenta Asterisk. 		& 2.0-1 \\
		\hline
		Asterisk 			& Software livre que permite a criação de PABX com diversos recursos. 					& 1.4 \\		
		\hline
		Asterisk-Java 		& Framework para comunicação com o Asterisk via protocolo AGI. 							& 1.0 \\
		\hline
		JUnit		 		& Framework para construção e execução de testes. 										& 4.0 \\		
		\hline			
	\end{tabular}\\[6pt]
	\fontsize{10}{12}\selectfont {Fonte: Autoria Própria.}
\end{table}



\section{\uppercase{Etapas da Integração}}
O processo de integração entre os sistemas está dividido em quatro etapas principais, das quais são necessárias para compor a solução escolhida, conforme definido abaixo e descrito nas sessões seguintes:

\begin{itemize}
	\item Implementação de \textit{WebServices} no Sistema GSAN. 
	\item Implementação do \textit{Middleware}.
	\item Customização do \textit{software} Asterisk.
	\item Desenvolvimento da suíte de testes automatizados.
\end{itemize}
