\section*{Aspectos de Inovação}
O trabalho de pesquisa e desenvolvimento se trata de uma integração entre sistemas totalmente distintos, com tecnologia \textit{Open Source}, onde juntos são capazes de atender à uma demanda existente no setor de saneamento brasileiro relacionado ao contexto do Atendimento ao Público.

Com a integração entre os sistemas GSAN e Asterisk, possibilitou a transferência dos atendimentos destinado a central de atendimento para uma Unidade de Resposta Audível \ref{key:URA}, que executa a triagem dos atendimentos de forma padronizada e organizada.

As solicitações referentes aos serviços Obter 2ª via de conta, Informar falta de água e Solicitar restabelecimento de ligação de água, são solucionados de forma automatizada em um fluxo pré-determinado na URA sem que haja intervenção humana durante o processo. 

Não há oficialmente uma versão publicada na comunidade do sistema GSAN capaz de atender esta demanda de forma igual ou superior, inovando em propor uma solução viável e eficiente de baixo custo, possibilitando a melhoria no atendimento ao cliente, se destacando em permitir que o mesmo possa realizar suas solicitações de forma rápida e padronizada em qualquer horário do dia ou da noite.