\section{Trabalhos Relacionados}
Este trabalho de pesquisa e desenvolvimento se assemelha ao trabalho descrito por Guilherme, (Vieira, 2007) que também utilizou recursos do programa Asterisk para desenvolver uma Sistema de criação de planos de discagem de forma prática explanando aspectos da ferramenta e expondo as dificuldades encontradas, apesar de ambos utilizarem dos diversos recursos do Asterisk, há divergência no objetivo onde este se destaca o fator de realizar uma integração com outro software, visando trazer uma solucionar uma demanda do setor de Saneamento.  No trabalho desenvolvido por Jilcimaico (DARÚ, 2008), aborda com clareza a utilização da distribuição Disc-OS como interface WEB do Asterisk, além de descreve os principais conceitos envolvidos na utilização do software, demonstra os procedimentos necessários para realizar a instalação da ferramenta e configuração dos recursos essências para um \textit{Call Center}, assemelhando-se este ao fato de também utilizar a distribuição Disc-OS que propõe uma interface WEB para a configuração do Asterisk.
O trabalho desenvolvimento por Humberto (CAMPOS, 2007), utilizou os principais recursos do software Asterisk para realizar uma integração com um sistema externo que calcula os valores de cada ligação realizada, módulo chamada de “tarifador” além de exibir os valores em um hardware próprio, tal integração utilizou como referência tabelas em banco de dados para reconhecer eventos ocorridos e ações a serem tomadas, assemelhando-se a este trabalho pelo fato de utilizar recursos do Asterisk para disparar ações de sistemas externos, no entanto a forma de integração retratada acima se diferencia da forma adotada neste trabalho, que utiliza o protocolo de interface de serviço AGI disponibilizado para comunicação com sistemas externos, o próprio Asterisk irá disparar ações a serem realizadas por meio deste protoloco.
Atualmente a empresa de saneamento Companhia Pernambucana de Saneamento (COMPESA, 2014) disponibilizou aos seus clientes o atendimento eletrônico por meio de URA, possibilitando a empresa realizar o atendimento destinado a central de atendimento, ou seja, o atendimento de primeiro nível, de forma automática e padronizada, propiciando também os direcionamentos entre ramais reais da empresa agilizando o atendimento e potencializando uma disponibilidade de 24 horas por 7 dias, com as informações a disposição dos clientes remotamente, porém a empresa não divulgou detalhes técnicos ou artefatos produzidos para realizar tal integração ou customização.
Para auxílio na elaboração deste trabalho de pesquisa se fez de grande valia os detalhes apontados sobre o \textit{software} Asterisk, principalmente a conceituação e protocolos disponibilizados para comunicação com sistemas externos, contidos no próprio \textit{website} da \textit{Digium}. 
