\section{Resultado dos Cenários de Teste}

Após a apuração dos cenários descritos acima, chegamos aos seguintes resultados por serviço:

\begin{table}[H]
	\center
	\footnotesize
	\caption{Avaliação do serviço 2ª via de Conta }
	\label{tabela:avaliacaoSegundaViaConta}
	\begin{tabular}{|p{3cm}|p{3cm}|p{3cm}|}
		\hline
		\multicolumn{3}{|c|}{\textbf{2ª via de Conta}} \\
		\hline
		\textbf{Cenários}  	& \textbf{Situação} & \textbf{Percentual (\%)}  \\
		\hline		
		Cenário 1			& Atendido 		& (+) 25\% 	\\
		\hline
		Cenário 2 			& Atendido		& (+) 25\% 	\\
		\hline
		Cenário 3 			& Atendido 		& (+) 25\%	\\
		\hline
		Cenário 4			& Não Atendido	& (-) 25\% 	\\
		\hline		
		\multicolumn{2}{|c|}{\textbf{TOTAL}}	& 75\% 	\\
		\hline				
	\end{tabular}
	\legend{\fontsize{10}{12}\selectfont {Fonte: Autoria Própria}.}
\end{table}

O serviço de obter 2ª via de conta obteve o percentual de 75\% de sucesso nos cenários de testes aplicados sobre o produto gerado, conforme demostra a tabela \ref{tabela:avaliacaoSegundaViaConta}. 


\begin{table}[H]
	\center
	\footnotesize
	\caption{Informar Falta de Água}
	\label{tabela:avaliacaoInformarFaltaAgua}
	\begin{tabular}{|p{3cm}|p{3cm}|p{3cm}|}
		\hline
		\multicolumn{3}{|c|}{\textbf{Informar Falta de Água}} \\
		\hline
		\textbf{Cenários}  	& \textbf{Situação} & \textbf{Percentual (\%)}  \\
		\hline		
		Cenário 1			& Atendido 		& (+) 25\% 	\\
		\hline
		Cenário 2 			& Atendido		& (+) 25\% 	\\
		\hline
		Cenário 3 			& Atendido 		& (+) 25\%	\\
		\hline
		Cenário 4			& Não Atendido	& (-) 25\% 	\\
		\hline		
		\multicolumn{2}{|c|}{\textbf{TOTAL}}	& 75\% 	\\
		\hline				
	\end{tabular}
	\legend{\fontsize{10}{12}\selectfont {Fonte: Autoria Própria}.}
\end{table}

O serviço de Informar Falta de Água obteve o percentual de 75\% de sucesso nos cenários de testes aplicados sobre o produto gerado, conforme demostra a tabela \ref{tabela:avaliacaoInformarFaltaAgua}.


\begin{table}[H]
	\center
	\footnotesize
	\caption{Solicitar Restabelecimento da Ligação}
	\label{tabela:avaliacaoRestabelerLigacaoAgua}
	\begin{tabular}{|p{3cm}|p{3cm}|p{3cm}|}
		\hline
		\multicolumn{3}{|c|}{\textbf{Solicitar Restabelecimento da Ligação}} \\
		\hline
		\textbf{Cenários}  	& \textbf{Situação} & \textbf{Percentual (\%)}  \\
		\hline		
		Cenário 1			& Atendido 		& (+) 25\% 	\\
		\hline
		Cenário 2 			& Atendido		& (+) 25\% 	\\
		\hline
		Cenário 3 			& Não Atendido 	& (-) 25\%	\\
		\hline
		Cenário 4			& Atendido		& (+) 25\% 	\\
		\hline		
		\multicolumn{2}{|c|}{\textbf{TOTAL}}	& 75\% 	\\
		\hline				
	\end{tabular}
	\legend{\fontsize{10}{12}\selectfont {Fonte: Autoria Própria}.}
\end{table}

O serviço de Solicitar Restabelecimento da Ligação obteve o percentual de 75\% de sucesso nos cenários de testes aplicados sobre o produto gerado, conforme demostra a tabela \ref{tabela:avaliacaoRestabelerLigacaoAgua}.



\section{Apresentação dos Resultados Obtidos}

Levando em consideração os resultados obtidos através da aplicação de todos os cenários de testes descritos acima sob os serviços desenvolvidos neste trabalho, observando os percentuais atingidos com base na média descrita dos principais Serviços do Atendimento ao Público, notamos que a solução proposta é capaz de reduzir em até 20,46\%, dos registros de atendimentos diários, conforme visto abaixo na figura \ref{figura:eficienciaServicos}:	


\begin{figure}[!htb]
	\centering
	\caption{Gráfico de avaliação dos serviços automatizados}
	\label{figura:eficienciaServicos}	
	\includegraphics{figuras/eficiencia_servicos.png}
	\legend {\fontsize{10}{12}\selectfont {Fonte: Autoria Própria}.}
\end{figure}

